%! Author = idegen
%! Date = 26/11/2021

\documentclass{article}
\usepackage[a4paper, total={6in, 8in}]{geometry}
\usepackage[dvipsnames]{xcolor}
\usepackage[utf8]{inputenc}
\usepackage[english]{babel}
\usepackage[default,oldstyle,scale=0.95]{opensans}
\usepackage[]{amsthm}
\usepackage[]{amssymb}
\usepackage{placeins}
\usepackage{textcomp}
\usepackage{setspace}
\usepackage{fancyhdr}
\usepackage[useregional]{datetime2}
\usepackage{hyperref}
\hypersetup{
    colorlinks=true,
    linkcolor=Emerald,
    filecolor=magenta,
    urlcolor=cyan,
    citecolor=RoyalBlue
}
\makeatletter
\def\namedlabel#1#2{\begingroup
#2%
\def\@currentlabel{#2}%
\phantomsection\label{#1}\endgroup
}
\makeatother

\setlength{\parindent}{0em}
\setlength{\parskip}{0.5em}

\pagestyle{fancy}
\fancyhf{}
\chead{Machine Learning Paradigms}
\rhead{\today}
\lhead{Isabella Degen}
\lfoot{isabella.degen@bristol.ac.uk}
\rfoot{Page \thepage}

\title{Machine Learning Paradigms - Coursework Report}
\author{Isabella Degen - Interactive AI CDT}

\begin{document}
    \maketitle


    \section{Introduction}\label{sec:introduction}
    Motivation and Approach %TODO: delete
    My goal for this coursework was to gain more experience with the various ML algorithms and evaluation methods we've
    seen during the semester, various different python frameworks available, methods to keep track of experiments, as
    well as learning more about how to apply good software engineering practices I'm used to from industry to a ML problem.
    I was less focused on finding the best solution or winning the competition.

    I followed the steps laid out on \href{https://www.kaggle.com/c/morebikes2021/overview/description}{Kaggle} and broke
    them down into the following sub steps:

    \subsection*{\label{phs:one}{Phase 1}}
    Predictions of station's 201-275 availabilities for one month by (a) training a model per station or (b)
    training one model for all stations.

    \begin{description}
        \item [\namedlabel{itm:phase1-step1}{Step 1}] Analyse and investigate station's test data and Decide on what problem it is. Classification (binary, multi label), Regression, ...
        \item [\namedlabel{itm:phase1-step2}{Step 2}] Experiment with a few models using approach a (training model per station)
        \item [\namedlabel{itm:phase1-step3}{Step 3}] Experiment with a few models using approach b (training one model for all station)
        \item [\namedlabel{itm:phase1-step4}{Step 4}] Tune the more promising potential solutions
        \item [\namedlabel{itm:phase1-step5}{Step 5}] Submit results
    \end{description}

    \subsection*{\label{phs:two}{Phase 2}}
    Use the pre-trained models for station 1-200 to predict availability for stations 201-275
    \begin{description}
        \item [\namedlabel{itm:phase2-step1}{Step 1}] Analyse given models
        \item [\namedlabel{itm:phase2-step2}{Step 2}] Experiment with a few approaches
        \item [\namedlabel{itm:phase2-step3}{Step 3}] Tune the more promising potential solutions
        \item [\namedlabel{itm:phase2-step4}{Step 4}] Investigate performance
        \item [\namedlabel{itm:phase2-step5}{Step 5}] Submit results
        \item [\namedlabel{itm:phase2-step6}{Step 6}] Compare with results from \nameref{phs:one}
    \end{description}

    \subsection*{\label{phs:three}{Phase 3}}
    Combine approaches from \nameref{phs:one} and \nameref{phs:two}
    \begin{description}
        \item [\namedlabel{itm:phase3-step1}{Step 1}] TODO

    \end{description}
    I keep referring back to these steps in section: \nameref{sec:method}, \nameref{sec:technical-background}, \nameref{sec:experiment-setup},
    \nameref{sec:results}. The final sections \nameref{sec:conclusions} gives a summary of my overall learnings and conclusions.

    %TODO reference to github
    %TODO reference to weight and biases graphs


    \section{Technical Background}\label{sec:technical-background}

    Be brief.

    Overall approach for engineering python environment setup: \cite{conda_forge_community_2015_4774216}, experiment tracking: \cite{wandb}

    \subsection*{\nameref{phs:one}}

    \subsubsection*{\nameref{itm:phase1-step1}}
    - Data Investigation in Notebook

    \section{Method}\label{sec:method}

    Description of method\\

    \subsection*{\nameref{phs:one}}

    \subsubsection*{\nameref{itm:phase1-step1}}
    Goal is to find out what's similar and what's different between the different station.
    Can I understand what the data means? What the different columns are

    From that decide what sort of problem it is:
    - Regression to predict a number but could be a classifier if we say the numbers are from a discrete set of numbers
    - Supervised learning given we've got labelled data
    - Might be worth to do some feature exploring with clustering to learn how the different features relate to each other

    \subsubsection*{\nameref{itm:phase1-step2}
    Given that the goal is to predict a number and given I have labelled data for some stations I decided that this is
    probably most easiest formulated as a regression problem. I want to not decide on the algorithm and instead try a few
    alorithms out before deciding on one.

    Before doing any more analysis I wanted to create the baseline for random guessing based on number of docks for each station.
    This also means I can quickly setup a pipline in pure python: load data, predict random numbers, calculate mean average error, log results
    and do this test driven. Further more all parameters are defined in a configurations class. This avoids magic nubmers and
    strings, ensures these values get logged in W&B which helps with reproducibility for each experiment even at this early stage.


    \section{Experiment Setup}\label{sec:experiment-setup}


    \section{Results}\label{sec:results}
    Results achieved


    \section{Conclusions}\label{sec:conclusions}
    share any insights you gained about how the system works.


    \section*{Acknowledgements}

    \bibliographystyle{plain}
    \bibliography{bibliography}
%TODO reference the varsious medium articles for software engineering
\end{document}
